\section{Decentralized search for \\ shortest path approximation}
\label{searching}

We propose to solve the point-to-point shortest path estimation problem using decentralized search with landmark based index. This section explain how decentralized search work with the index and underlying graph. Several aspects of the search including termination criterion, bidirectional search and tie breaking strategy are also discussed. 

\begin{figure*}[t]
    \centering
    \subfigure[Decentralized search]{\epsfig{file=./figures/new_illustrate/dec_common.pdf,width=0.32\textwidth}}
    \subfigure[Bi-directional Decentralized Search]{\epsfig{file=./figures/new_illustrate/bi_dec.pdf,width=0.32\textwidth}}
    \subfigure[Tie Strategy]{\epsfig{file=./figures/new_illustrate/tie.pdf,width=0.32\textwidth}}
    \caption{Cumulative distribution of the number of terminal nodes sampled by different algorithms.}
    \label{fig:cum_dis}
\end{figure*}

\subsection{Index guided decentralized search}

We perform decentralized search on an indexed graph as follows. For a given pair of source and target vertex, the search start at the source vertex append it to the approximated path. The search examines each neighbor of vertex currently visited, for each neighbor, the LCA distance to the target vertex is calculated. The neighbor with least LCA distance will be picked as the vertex to visit next and append it to the approximated path. The search continues until it reach the target vertex. 

However, terminating when the search reaches the target vertex is a valid but not an ideal terminate criteria. Since the label of each vertex stores the shortest path of each vertex to each landmark. And shortest path follows the optimal substructure, i.e. the path between any two vertices along the shortest path is also the shortest path of them. So a search can stop once it  reaches any vertex in the label of the target vertex. The path contained in the label of target vertex is already the the shortest path due to the optimal substructure, we can directly concatenate it to the visited vertices to form a approximated path. 

The detailed algorithm of decentralized search is depicted in \ref{alg:dec}.

\begin{algorithm}
    \caption{Algorithm decentralized search}
		\label{alg:dec}
    \begin{algorithmic}
        \Function{DecentralizedSearch}{$G$, $s$, $t$}
					\State $p \gets \emptyset$
					\State $u \gets s$
					\State $p = p \cup u$
					\While{$u \notin L(t)$}
						\State $d_{min} \gets \infty$
						\State $w \gets u$
						\For{each $v$ adjecent to $u$}
							\If{$d_{LCA}(v,t) < d_{min}$}
								\State $d_{min} \gets d_{LCA}(V,t)$
								\State $w \gets v$ 
							\EndIf
						\EndFor
						\State $u \gets w$
						\State $p = p \cup u$
					\EndWhile
					\State $p_{remain} \gets$ path from $u$ to $t$ in $L(t)$ exclude $u$
					\State $p = p \cup p_{remain}$
					\State \Return $p$
        \EndFunction
    \end{algorithmic}
\end{algorithm}

At each step, by examining neighbor vertices the search is able to explore large part of the graph that is not included in the label of source and target vertex, which can potentially increase both accuracy and diversity of the path being found. For example in Fig. \ref{fig:dec_common} from $15$ to $11$, by following the procedure of decentralized search, instead of finding a short cut edge with both ends in labels of vertex $15$: $(0, 3, 6, 9, 15)$ and $11$: $(0, 2, 11)$, decentralized search can find a edge $(9, 4)$ that can lead to a shorter path which is denoted by solid curved line. 

So will the index guided decentralized search eventually terminate? As long as the source vertex $s$ and target vertex $t$ are reachable from each other, decentralized search will terminate in as much as ${\sigma}_{max}$ steps, where ${\sigma}_{max}$ is the diameter of the graph. To see this, first, the distance from $s$ to any vertex in $L(v)$ is shorter than ${\sigma}_{max}$. And at each step, suppose decentralized search is visiting vertex $u$, there must be a neighbor vertex $v$ that is on the path indicated by LCA computation of $u$ and $t$ that meet $d_{LCA}(v,t) \leq d_{LCA}(u,t) - 1$. Since decentralized search always pick the neighbor with least LCA distance to the target, the LCA distance to the target at each step will decrease at least by $1$. Therefore, decentralized search will terminate in at most ${\sigma}_{max}$ steps. 

The time complexity of Decentralized search depends on the max degree and the diameter of the graph. Decentralized search take at most ${\sigma}_{max}$ steps to finish. For each step, the search need to check at most ${\delta}_{max}$ neighbor vertices, where ${\delta}_{max}$ is the max vertex degree of the graph. For each neighbor, $k$ LCA computations are required where $k$ is the number of landmarks. The compuational time required for each LCA computation is $O(h)$ where $h$ is the height of the indexed shortest path tree. And we have $h \leq {\sigma}_{max}$ So the worst case time complexity of decentralized search is $O(k{{\sigma}_{max}}^2{\delta}_{max})$. 

The space complexity for decentralized search contains two parts, offline index space complexity and online query space complexity. The space required for offline index is $O(k{\sigma}_{max}n)$. For each query, $O(k{\sigma}_{max})$ space is required to store the labels of target vertex and the vertex that is being examined, $O({\sigma}_{max})$ space is required to store the approximated path. Combining them together, the online search space complexity of decentralized search is $O(k{\sigma}_{max})$.

\subsection{Bi-directional search}

In this section we show how to combine the bidirectional search and decentralized search together. Unlike bidirectional BFS, the goal for performing bidirectional search is not to reduce search space but to increase accuracy. A reversed search starting at target vertex and aim at source vertex may explore a different set of vertices and edges which may lead to a different path, sometimes with shorter length, to be found compared to the original search. For example in Fig. \ref{fig:bi_dec}, the search starts from $20$ to $7$ can find a shorter path $p = (20, 17, 19, 5, 7)$ than the search starts at $7$. Due to that $0$ has a smaller LCA distance to $7$ than $19$, the edge $(19, 5)$ cannot be found by the search starts from $20$. 

In BFS, we expect two search will meet in some intermedia vertices that can be used as a new stop criterion which lead to reduced search space. But in decentrailzed search, there is no guarantee that two search will meet at any vertex except the source/target vertex. Since two search are driven by two distinct goals: finding next hop with least LCA distance to source/target vertex. 

[explained by an example]

\subsection{Handle ties}

Tie happens frequently in decentralized search especially when the number of landmark is small. The tie here means during a step when decentralized search examining neighbors of currently visited vertex, there is not sufficient information in the index that can separate several neighbors, i.e. they have the same LCA distance to the target. For example in Fig. \ref{fig:tie}, to find path from $8$ to $6$, when traversing neighbors of vertex $8$, both vertex $11$ and $4$ have the same LCA distance to vertex $6$, but their actual distances to vertex $6$ are different due to edges currently invisible to the decentralized search. Labels of each neighbor, on the other hand, provide no clue which one can lead to a shorter path.

Expanding the search onto each tie vertex require the search examine different sets of vertices and edges which will increase the cost of the search, but will increase the chance to find a shorter path as well. So two obvious solution to deal with ties are either randomly pick one vertex to visit or visit all vertices in the next step. The former one incur no additional cost and will have the least possibility to find a shorter path. The latter one require most effort and will lead to the shortest path the decentralized search could find.

[the lca heuristic and effort/shorter path ratio.]
