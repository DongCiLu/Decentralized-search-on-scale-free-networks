\section{Introduction}
\label{intro}

Introduced by the famous experiment conducted by Stanley Milgram, the small-world phenomenon, that most vertices can be reached from each other by a small number of hops, is found to be an fundamental characteristic in most real world networks. Although this property assure us the existence of such short paths between vertices in small world networks, due to the ever increasing number of vertices, this seemingly straightforward operation has become challenging. 
For example, for graph derived from LiveJournal social networks with 4 million vertices and 50 million edges \cite{Backstrom:2006:GFL:1150402.1150412}, online BFS/Dijkstra traversals will take around a minute to finish on a single commercial computer, even for A* search which can significantly reduce the number of vertices visited, it still require to access an average of 20K vertices for each query \cite{Potamias:2009:FSP:1645953.1646063}. 

Various networks which are commonly used in modern information technologies to model real-world phenomenon, such as online social networks (e.g., LinkedIn or Facebook), biological networks, road networks, etc have millions of vertices and billions of edges that even a cluster of machines can hardly deal with them. 
For example, in Figure~\ref{fig:trend}, we plot the runtime of BFS running on a Amazon EC2 cluster with 10 machines for networks of various size. Networks are derived from Facebook, Twitter and LiveJournal respectively \cite{Mcauley:2014:DSC:2582178.2556612}, \cite{Backstrom:2006:GFL:1150402.1150412}. We can see that even for a cluster of machines, a single BFS traversal takes around 3.5 seconds, and the time required for the traversal increases quickly as the scale of the network increases.

\begin{figure}[t]
    \centering
    \includegraphics[width = 3in]{../figures/BFS_compare.pdf}
    \caption{This figure shows the runtime of a single BFS on a Amazon EC2 cluster with 10 machines for graphs of various size.}
    \label{fig:trend}
\end{figure}

In this review, we are focusing on one of the most common variants of the shortest path/distance problem, to find a point-to-point shortest path/distance in a graph. This operation serves as the building block for many other tasks, and has many applications. For example, a natural application for road network is providing driving directions \cite{Abraham:2011:HLA:2008623.2008645}. In social networks, such applications include social sensitive search \cite{Vieira:2007:ESR:1321440.1321520}, analyze influential people \cite{Kempe:2003:MSI:956750.956769}. Estimating minimum round trip time between hosts without direct measurement is another application in technology networks \cite{Tang:2003:VLI:948205.948223}.

Usually in these applications, point-to-point shortest path/distance problem will be solved repeatedly for different source/target vertex pairs. A straight forward solution to this problem is to apply online traversal like BFS/Dijkstra for each query. But clearly, this approach is very inefficient due to the scalability issue discussed above. Due to the nature of applications' needs, most point-to-point shortest path/distance algorithms will use preprocessing in order to speed up the following online searching operations. One obvious way for preprocessing is to perform all-pair shortest paths operation such as Floyd-Warshall and store the results for each pair of vertices. When a query arrives, only constance time is required to return the results. However, this approach is also unacceptable due to the $O(V^3)$ runtime and $O(V^2)$ storage space. To balance between preprocessing and online query has become a key challenge in this research area \cite{Potamias:2009:FSP:1645953.1646063}, \cite{tretyakov2011fast}, \cite{Akiba:2012:SQC:2247596.2247614}, \cite{6399472}, \cite{Jin:2012:HLA:2213836.2213887}.

The point-to-point shortest path/distance problem has already been well studied in road networks, distance queries can be answered in less than a microsecond for the complete USA road network \cite{Abraham:2011:HLA:2008623.2008645}. Most of the methods on road networks take advantage of the spatial and planar-like properties a road network has \cite{Gubichev:2010:FAE:1871437.1871503}. Especially, Abraham et al. discovered that several fastest distance computation algorithms actually rely on a graph measure called highway dimension. However, these existing algorithms for road networks can hardly be applied to complex networks, due to the fact that the structure of most complex networks is nowhere near planar, and they usually do not have low highway dimensions like road networks. Actually, the locality of real-world complex networks is very poor, since the number of edges crossing clusters of the network is very large \cite{Leskovec08communitystructure}.

To further increase the scalability to handle large-scale complex networks, approximate methods have also been studied. Instead of returning an exact shortest path/distance between query vertex pairs, a near optimal path/distance is returned. Compared to online traversals for finding exact shortest path in seconds, it usually takes milliseconds for an algorithm to return an approximated shortest path \cite{Gubichev:2010:FAE:1871437.1871503}, \cite{tretyakov2011fast}, and microseconds to approximate shortest path distance \cite{Akiba:2013:FES:2463676.2465315}.

\subsection{Problem Formulation}

Our problem formulation is as follows: we investigate how to find exact or approximate point-to-point shortest paths/distance between any two vertices in extremely large networks. By extremely large, we are concerned with networks of million vertices, but in practice, such networks may even contain hundreds of millions of vertices. We are focusing on a specific kind of networks which is very common in real-world networks called complex network. They usually do not exhibit the same properties as simple ones such as road networks. For example, it is well known that these networks have their vertices' degrees conforming to the power law, and that vertices have relatively short distances between themselves. In most cases, the locality of these networks are very poor.

\subsection{Contributions}

In this review, we provide a comprehensive state-of-the-art study in algorithms on point-to-point shortest path/distance problems in large-scale complex networks. Existing algorithms usually contain two phases, distance calculation and online searching. After preprocessing, algorithms can return exact or estimated shortest path distance of two given vertices that can be used in applications where the underlying path is not important. The later one return an actual exact or approximate shortest path, and is essential for several applications require path information. Usually, online searching needs distance estimates to avoid large number of visited vertices, which is the main benchmark of these kind of algorithms.

\subsection{Paper organization}

The remaining of this review is organized as follows. In Section~\ref{distance}, we present several fundamental preprocessing methods being used by most of the algorithms and describes algorithms that calculate exact or estimated shortest path distance for any given vertex pair. We discuss online searching algorithms that finding actual paths in Section~\ref{path}. We conclude the whole review in Section~\ref{conclusion}.
