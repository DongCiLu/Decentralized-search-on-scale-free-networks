% This is "sig-alternate.tex" V2.1 April 2013
% This file should be compiled with V2.5 of "sig-alternate.cls" May 2012
%
% This example file demonstrates the use of the 'sig-alternate.cls'
% V2.5 LaTeX2e document class file. It is for those submitting
% articles to ACM Conference Proceedings WHO DO NOT WISH TO
% STRICTLY ADHERE TO THE SIGS (PUBS-BOARD-ENDORSED) STYLE.
% The 'sig-alternate.cls' file will produce a similar-looking,
% albeit, 'tighter' paper resulting in, invariably, fewer pages.
%
% ----------------------------------------------------------------------------------------------------------------
% This .tex file (and associated .cls V2.5) produces:
%       1) The Permission Statement
%       2) The Conference (location) Info information
%       3) The Copyright Line with ACM data
%       4) NO page numbers
%
% as against the acm_proc_article-sp.cls file which
% DOES NOT produce 1) thru' 3) above.
%
% Using 'sig-alternate.cls' you have control, however, from within
% the source .tex file, over both the CopyrightYear
% (defaulted to 200X) and the ACM Copyright Data
% (defaulted to X-XXXXX-XX-X/XX/XX).
% e.g.
% \CopyrightYear{2007} will cause 2007 to appear in the copyright line.
% \crdata{0-12345-67-8/90/12} will cause 0-12345-67-8/90/12 to appear in the copyright line.
%
% ---------------------------------------------------------------------------------------------------------------
% This .tex source is an example which *does* use
% the .bib file (from which the .bbl file % is produced).
% REMEMBER HOWEVER: After having produced the .bbl file,
% and prior to final submission, you *NEED* to 'insert'
% your .bbl file into your source .tex file so as to provide
% ONE 'self-contained' source file.
%
% ================= IF YOU HAVE QUESTIONS =======================
% Questions regarding the SIGS styles, SIGS policies and
% procedures, Conferences etc. should be sent to
% Adrienne Griscti (griscti@acm.org)
%
% Technical questions _only_ to
% Gerald Murray (murray@hq.acm.org)
% ===============================================================
%
% For tracking purposes - this is V2.0 - May 2012

\documentclass{sig-alternate-05-2015}

\usepackage{amsmath}
\usepackage{algorithm}
\usepackage[noend]{algpseudocode}
\usepackage[flushleft]{threeparttable}
\usepackage{subfigure}

\begin{document}

%% Copyright
%\setcopyright{acmcopyright}
%%\setcopyright{acmlicensed}
%%\setcopyright{rightsretained}
%%\setcopyright{usgov}
%%\setcopyright{usgovmixed}
%%\setcopyright{cagov}
%%\setcopyright{cagovmixed}
%
%
%% DOI
%\doi{10.475/123_4}
%
%% ISBN
%\isbn{123-4567-24-567/08/06}
%
%%Conference
%\conferenceinfo{PLDI '13}{June 16--19, 2013, Seattle, WA, USA}
%
%\acmPrice{\$15.00}
%
%%
%% --- Author Metadata here ---
%%\conferenceinfo{WOODSTOCK}{'97 El Paso, Texas USA}
%%\CopyrightYear{2007} % Allows default copyright year (20XX) to be over-ridden - IF NEED BE.
%%\crdata{0-12345-67-8/90/01}  % Allows default copyright data (0-89791-88-6/97/05) to be over-ridden - IF NEED BE.
%% --- End of Author Metadata ---

\title{Decentralized Search for Shortest Path Approximation \\
in Large-scale Complex Networks}
%\subtitle{[Extended Abstract]
%\titlenote{A full version of this paper is available as
%\textit{Author's Guide to Preparing ACM SIG Proceedings Using
%\LaTeX$2_\epsilon$\ and BibTeX} at
%\texttt{www.acm.org/eaddress.htm}}}
%
% You need the command \numberofauthors to handle the 'placement
% and alignment' of the authors beneath the title.
%
% For aesthetic reasons, we recommend 'three authors at a time'
% i.e. three 'name/affiliation blocks' be placed beneath the title.
%
% NOTE: You are NOT restricted in how many 'rows' of
% "name/affiliations" may appear. We just ask that you restrict
% the number of 'columns' to three.
%
% Because of the available 'opening page real-estate'
% we ask you to refrain from putting more than six authors
% (two rows with three columns) beneath the article title.
% More than six makes the first-page appear very cluttered indeed.
%
% Use the \alignauthor commands to handle the names
% and affiliations for an 'aesthetic maximum' of six authors.
% Add names, affiliations, addresses for
% the seventh etc. author(s) as the argument for the
% \additionalauthors command.
% These 'additional authors' will be output/set for you
% without further effort on your part as the last section in
% the body of your article BEFORE References or any Appendices.

%\numberofauthors{3} %  in this sample file, there are a *total*
%% of EIGHT authors. SIX appear on the 'first-page' (for formatting
%% reasons) and the remaining two appear in the \additionalauthors section.
%%
%\author{
%% You can go ahead and credit any number of authors here,
%% e.g. one 'row of three' or two rows (consisting of one row of three
%% and a second row of one, two or three).
%%
%% The command \alignauthor (no curly braces needed) should
%% precede each author name, affiliation/snail-mail address and
%% e-mail address. Additionally, tag each line of
%% affiliation/address with \affaddr, and tag the
%% e-mail address with \email.
%%
%% 1st. author
%\alignauthor Zheng Lu\\
       %\affaddr{Electrical Engineering and Computer Science}\\
       %\affaddr{University of Tennessee, Knoxville}\\
       %\email{zlu12@vols.utk.edu}
%% 2nd. author
%\alignauthor Yunhe Feng\\
       %\affaddr{Electrical Engineering and Computer Science}\\
       %\affaddr{University of Tennessee, Knoxville}\\
       %\email{yfeng14@vols.utk.edu}
%% 3rd. author
%\alignauthor Qing Cao\\
       %\affaddr{Electrical Engineering and Computer Science}\\
       %\affaddr{University of Tennessee, Knoxville}\\
       %\email{cao@utk.edu}
%}

\date{30 April 2016}


\maketitle
\begin{abstract}
Finding approximate shortest paths for extremely large-scale complex networks is still a challenging problem, where existing work requires too much overhead to achieve accurate results and good path diversity for graphs with up to billions of edges. In this paper, we develop a online search method based on preprocessed indexes, to approximate shortest path of arbitrary pairs of vertices. We demonstrate that the algorithm achieves much higher accuracy and path diversity with less index overhead compared to previous work. The algorithm can also achieve differentiated level of accuracies by dynamically controlling the search space to meet various application needs. In order to perform decentralized search more efficiently, we propose a new heuristic index construction algorithm that can greatly increase the approximation accuracy. We further develop support for parallel searches to further reduce the average search time for large amount of queries. We implement our algorithm in distributed settings to deal with extreme size graphs. Our algorithm can handle graphs with billions of edges and perform searches for millions of queries in parallel. We evaluate our algorithm on various real world graphs from different disciplines.
\end{abstract}


%
% The code below should be generated by the tool at
% http://dl.acm.org/ccs.cfm
% Please copy and paste the code instead of the example below. 
%
\begin{CCSXML}

<ccs2012>


<concept>

<concept_id>10002951.10002952.10002971</concept_id>
 <concept_desc>Information systems~Data structures</concept_desc>

<concept_significance>500</concept_significance>
</concept>

<concept>

<concept_id>10002951.10003227.10003351</concept_id>
 <concept_desc>Information systems~Data mining</concept_desc>

<concept_significance>300</concept_significance>
</concept>
</ccs2012>

\end{CCSXML}

\ccsdesc[500]{Information systems~Data structures}
\ccsdesc[300]{Information systems~Data mining}

%
% End generated code
%

%
%  Use this command to print the description
%
\printccsdesc

% We no longer use \terms command
%\terms{Theory}

\keywords{Graphs, Shortest Paths, Decentralized Search}

\section{Introduction}
\label{introduction}

Various types of graphs are commonly used as models for real-world phenomenon, such as online social networks, biological networks, the world wide web, among others~\cite{newman2010networks}. As their sizes keep increasing, scaling up algorithms to handle extreme size graphs with billions of edges remains a challenge that has drawn increased attention in recent years. Specifically, straightforward graph algorithms are usually too slow or costly when they are applied to graphs at this scale. One problem is finding shortest paths in the network, an operation that serves as the building block for many other tasks. For example, a natural application for road network is providing driving directions~\cite{Abraham:2011:HLA:2008623.2008645}. In social networks, such applications include social sensitive search~\cite{Vieira:2007:ESR:1321440.1321520}, analyzing influential people~\cite{Kempe:2003:MSI:956750.956769}. Estimating minimum round trip time between hosts without direct measurement is another application in technology networks~\cite{Tang:2003:VLI:948205.948223}.

Although previous works have studied the shortest path problem on large road networks extensively. A large category of networks known as the complex networks has very different structures, i.e. following power law degree distributions, exhibiting small diameters, etc., and has received less attentions. Approaches for road networks do not perform well on complex networks. In this paper, we focus on the shortest path problem for complex networks in particular, as their extreme sizes and unique topologies make the problem particularly challenging.

Our design is motivated by recent studies that combine both offline processing and online queries~\cite{Potamias:2009:FSP:1645953.1646063, tretyakov2011fast, Akiba:2012:SQC:2247596.2247614, 6399472, Jin:2012:HLA:2213836.2213887}. In these methods, the step of preprocessing aims to construct indexes for the networks, which are later used in the online query phase to dramatically reduce the query time. Among these approaches, landmark based algorithms are widely used to approximate shortest path/distance between vertices~\cite{Thorup:2005:ADO:1044731.1044732, Goldberg:2005:CSP:1070432.1070455, Potamias:2009:FSP:1645953.1646063, Gubichev:2010:FAE:1871437.1871503, tretyakov2011fast, 6399472}. Such algorithms select a small set of landmarks, and construct an index that consists of labels for each vertex, which store distances or shortest paths to landmarks. The approximation accuracy of landmark based algorithms heavily depends on the number of landmarks. To achieve high accuracy, a relatively large set of landmarks is required, which leads to large preprocessing overhead. Indexes that can answer path queries usually have much larger space overhead than indexes that can only answer distance queries. One goal of our design, therefore, is to provide accurate results while still maintain low overhead for indexing.

Previous works on applying online search to indexed graph limit the search space to sub-graphs constructed by vertices in labels of source and target vertices~\cite{Gubichev:2010:FAE:1871437.1871503, 6399472}. The accuracy and diversity of approximated path are constrained this way, e.g., only short-cut edges directly connecting vertices in labels can be found. To overcome this problem, we propose to perform a heuristic search on the indexed graph that is guided by locally collected information from labels of nearby vertices. The advantage is that the search can expand the search space into edges that have not been indexed to achieve higher accuracy and diversity of approximated path with limited index size. The heuristic search that we use is called decentralized search which was introduced in~\cite{Kleinberg:2000p5066, kleinberg2006complex}. Here the ``decentralized'' means that the decision of the search is made based solely on local information which, in our context, is the labels of neighbor vertices at each step of the search.

Decentralized search is very light-weighted. The number of visited vertices for decentralized search is bounded by the diameter of the network. Considering that complex networks usually have relatively short diameters, decentralized search can finish in a limited number of steps. The search can also adjust its search space to balance between different levels of performance and required resources for each search. This makes the search very versatile to meet various application needs.  

The performance of decentralized search relies heavily on the index. Landmark selecting problem has been well studied in ~\cite{Potamias:2009:FSP:1645953.1646063,6927522}. We observe that even with the same landmark set, choosing which shortest path from vertex to landmark to be indexed also plays an important role for the accuracy of online search. To achieve better accuracy without increasing index overhead, we introduce a heuristic index construction algorithm to control shortest paths to be indexed during preprocessing. The proposed approach outperforms random shortest path indexing by a large margin on real networks.

Based on our algorithm design, we further develop a query-processing system based on distributed cloud infrastructure to support large scale graph with billions of edges. In this platform, users first submit their graphs for preprocessing needs. The graph processing engine will assign resources according to application's need for accuracy and construct an index for the input graph. Later, users may submit large volumes of queries repeatedly, for which responses will be generated. Applications that generate queries (on the client side) can provide their desired accuracy levels and the graph processing engine can dynamically adjust search space of decentralized search to meet differentiated levels of accuracies.

The light-weighted decentralized search allows a large number of queries to run in parallel so that the system can achieve high query processing throughput. There are two properties of decentralized search that makes it very suitable for parallel processing. First, decentralized search has small space complexity and communication complexity. As the search does not need to store any information on per-vertex basis like BFS or A* search, very limited space overhead is required for each search. Second, decentralized searches only have read after read data dependencies on the index and underlying graph. Multiple searches can run independently on the same graph and index. These two properties make it possible for a large number of searches running in parallel efficiently without reaching the physical limit of machines, i.e., memory size or network bandwidth. For example, in our experiments, we show that millions of decentralized search can run in parallel on graphs with billions of edges on a cluster of commodity machines, and finish in tens of seconds. 

\subsection{Contributions}
Our contributions can be summarized as follows:

\begin{itemize}
	\item We propose index guided decentralized search for shortest path approximation;
	\item We design a heuristic index construction algorithm to improve online search accuracy without increasing index overheads;
	\item We achieve efficient query processing and good scalability with distributed implementation and parallel processing;
	\item Experiments on various real world complex networks demonstrate that the proposed algorithm is promising in approximating shortest path compared to existing works.
\end{itemize}

The rest of this paper is organized as follows. In Section~\ref{relatedwork} we show previous works on exact and approximate approaches. Section~\ref{preliminary} provides notations and definitions used in this paper. We explain index guided decentralized search for shortest path approximation in Section~\ref{searching}. Section~\ref{preprocessing} discusses index construction algorithm. In Section~\ref{implementation} we show details on our distributed implementation. The evaluations of our algorithm is in Section~\ref{evaluation}. We conclude our work in Section~\ref{conclusion}.

\section{Related Works}
\label{relatedwork}

Existing work on shortest path/distance can be classified into exact approaches and approximate approaches. 

\textbf{Exact Approaches:} Majority of the exact approaches are based on either 2-hop cover \cite{Cohen:2002:RDQ:545381.545503}, \cite{Akiba:2013:FES:2463676.2465315} or tree decomposition \cite{Akiba:2012:SQC:2247596.2247614}, \cite{Wei:2010:TES:1807167.1807181}. For the former one, finding optimal 2-hop covers is a challenging problem. \cite{Akiba:2013:FES:2463676.2465315} takes a different approach that solving 2-hop cover problem with graph traversals which has better scalability. \cite{Jin:2012:HLA:2213836.2213887} borrowed the highway concept from shortest path algorithms on road networks and construct a spanning tree as a "highway". \cite{Fu:2013:IIB:2536336.2536346} introduced an effective disk-based label indexing method based on independent set.

\textbf{Approximate Approaches:} Since the exact approaches do not scale well, approximate algorithms are also well studied for large-scale complex networks. Landmark based algorithms are extensively studied for approximating shortest path/distance \cite{Thorup:2005:ADO:1044731.1044732}, \cite{Goldberg:2005:CSP:1070432.1070455}, \cite{Potamias:2009:FSP:1645953.1646063}, \cite{floreskul2014memory}, \cite{Maier:2011:INS:1993077.1993079}. Although theoretical study of such algorithms does not reveal promising results \cite{Thorup:2005:ADO:1044731.1044732}, they work well in practice. \cite{Potamias:2009:FSP:1645953.1646063}, \cite{6927522} studied various landmark selection strategies for constructing better indexes. A common problem for distance-only indexes is that they do not perform well for close pairs of vertices \cite{Akiba:2012:SQC:2247596.2247614}. Algorithms \cite{Gubichev:2010:FAE:1871437.1871503}, \cite{tretyakov2011fast}, \cite{6399472} which index shortest paths are proposed to alleviate this problem. Beside landmark based approaches, there are several other approximate approaches. \cite{7004250} forms the shortest path problem as a learning problem to predicting pairwise distance. \cite{Zhao:2010:OSP:1863190.1863199} maps vertices to low-dimension Euclidean coordinate spaces to answer distance queries in constant time.

\textbf{Combining online search with indexes:}
\cite{Goldberg:2005:CSP:1070432.1070455} uses A* search for online query based on indexes constructed by landmark based algorithms. However, the cost of each A* search is still very high for large scale networks. \cite{Gubichev:2010:FAE:1871437.1871503} perform BFS on a sub-graph generated by the labels of source and target vertex. Although search space is greatly reduced, it still needs around seconds to handle graphs with millions of vertices.

\section{Preliminaries}
\label{preliminary}

In our problem, we consider a graph $G = (V,E)$. For a source node $s$ and a target node $t$, we are interested in finding a path $p(s,t)=(s,v_1,v_2,...,t)$ with a length of $|p(s,t)|$ close to the exact distance $d_G(s,t)$ between $s$ and $t$. We focus on unweighted, undirected graphs in this paper.

Our method is motivated by the idea of using landmarks as the basis for indexes. Specifically, given a graph $G$ and a set of $k$ landmarks $(l_1,l_2,...,l_k)$, an index contains a label $L(v)$ for each vertex that stores the shortest path to each landmark. The label can be constructed by building a shortest path tree $SPT$ using BFS from each landmark.

The least common ancestor of two vertices in a tree is the farthest ancestor from the root, which we denote as $LCA(s,t)$. The shortest distance satisfies the triangle inequality, i.e., for an arbitrary pair of vertices $s$ and $t$, the following bound holds:
\begin{equation}
\label{equ:upper}
    d_G(s,t) \leq min_{l}\{d_G(s,LCA_{l}(s,t)) + d_G(LCA_{l}(s,t),t)\}
\end{equation}
This upper bound, which is referred to as the LCA distance and denoted by $d_{LCA}(s,t)$, can be used as an approximation of the distance from $s$ and $t$. We denote the path indicated by this distance as $p_{LCA}(s,t)$. The LCA distance and related path for a specific landmark $l$ is denoted as $d_{LCA_l}(s,t)$ and $p_{LCA_l}(s,t)$ respectively.
\section{Decentralized search for \\ shortest path approximation}
\label{searching}

In this section, we are going to discuss how to perform decentralized searches based on the index structures to achieve a higher accuracy. We will also discuss how to optimize and control the search space of decentralized search to maintain low online search overheads.

%\subsection{Source of error for index-only approximation}
%
%\begin{figure}[t]
    %\centering
    %\includegraphics[width=\linewidth]{../figures/new_illustrate/sc_illustrate.pdf}
    %\caption{Edges that do not exist in index cannot be explored while estimating shortest paths solely from labels of source and target vertices. Bold lines represent edges in index.}
    %\label{fig:sc_illustrate}
%\end{figure}
%
%Landmark based algorithms only encode a small portion of edges into the index. Due to that approximated paths only contains edges existing in the index, other edges become the source of approximation errors. For example in Fig. \ref{fig:sc_illustrate}, assuming vertex $0$ is the landmark, bold lines represent edges that have been encoded in the index. Although vertex $4$ and $9$ are adjacent in the underlying graph, the edge $(4,9)$ does not exist in the index. So the path estimated solely by the index $(4, 2, 0, 3, 6, 9)$ will not reflect the true distance of vertex $4$ and $9$. Same problem happens with vertex $5$ and $19$. Even vertices that are not directly connected with these edges may also be effected. For example, from vertex $6$ to $18$, shortest path $(6, 3, 12, 7, 18)$ cannot being extracted directly from the index due to both $(3, 12)$ and $(12, 7)$ are not in the index. Even though we are dealing with sparse graphs, number of edges that have been encoded in index for each landmark is less than the number of vertices, and number of vertices is usually much less than total number of edges. Thus a lot of edges are not encoded in the index. Simply increasing the number of landmarks will encode more edges into the index, which will lead to more accurate results, but also bringing more overheads at same time.

\subsection{Index guided decentralized search}

\begin{figure}[t]
    \centering
    \includegraphics[width=\linewidth]{../figures/new_illustrate/basic_dec.pdf}
    \caption{Decentralized search finds edges that are invisible to labels of the source and target vertices. The dashed curved line shows the path found by LCA distance. The solid curved line shows the path found by the decentralized search.}
    \label{fig:basic_dec}
\end{figure}

By indexing the shortest paths from each landmark in the label of each vertex, we are able to answer both distance and path queries by simple LCA computations with labels of source and target vertices. However, paths returned by LCA computations only contain edges that already exist in the index structure. Observe that, on the other hand, only relatively a small part of the network, i.e., the edges along the shortest path tree, has been indexed. Therefore, such computations may not lead to a high accuracy. For example, in Fig. \ref{fig:basic_dec}, using LCA distances to estimate the shortest path from $18$ to $16$ will return an approximated path $p = (18, 7, 5, 1, 12, 16)$, as represented by the dashed curved line. Note that there is an edge $(7,12)$ that has not been indexed. Therefore, the path $p = (18,7,12,16)$ cannot be found by computing LCA distance directly from labels of vertex $18$ and $16$.

To overcome such difficulties, we use a decentralized search in our design, which, at each step, will examine all the neighbors of current visited vertex and select one that is closest to the target. The search will continue this procedure at each step until it reaches the target vertex. In our case, the distance to the target is estimated by the LCA distance. By calculating the LCA distance to the target from each neighbor, the search can explore edges that do not exist in the labels of source and target vertices. For example, if we use decentralized search in Fig. \ref{fig:basic_dec} to estimate shortest path from $18$ to $16$, when examining neighbors of vertex $7$, vertex $12$ with a LCA distance of $1$ to the target will be selected for next step instead of vertex $5$ with a LCA distance of $3$. In this way, a shorter path can be found by decentralized search through examining edge $(7,12)$ which is not indexed in the labels of source and target vertex.

%Similar to A* search, decentralized search need information on how far each neighbor to the target to determine which vertex to pick as next hop for each step. The difference of decentralized search and A* search is that it makes the decision based solely on local information. By local we mean at each step, decentralized search will only examine the neighbors of currently visited vertex. So all the vertices visited by decentralized search will be a part of the estimated path.

\begin{figure}[t]
    \centering
    \includegraphics[width=\linewidth]{../figures/new_illustrate/dec_common.pdf}
    \caption{Decentralized search explores the edges that are not directly connected to the indexed shortest path. Edge $(4, 9)$ cannot be found by solely searching circles in the path by LCA distance.}
    \label{fig:dec_common}
\end{figure}

Next, observe that the edge $(7,12)$ shown in Fig. \ref{fig:basic_dec} can also be found by searching if there is any edge existing between any pair of vertices on path returned by LCA computation, so that we can avoid traversing neighbors of any vertex. But this could only find edges that have both ends in the labels of the source and target vertex, which is a very small subset of all the edges. Rather than constraining the search space in this way, decentralized search examines all edges adjacent to current visited vertex, which is a more reasonable and larger subset of edges to examine. For example in Fig. \ref{fig:dec_common} from $15$ to $11$, instead of finding a edge with both ends in labels of vertex $15$: $(0, 3, 6, 9, 15)$ and labels of $11$: $(0, 2, 11)$, decentralized search can find a edge with only one end in label vertex $15$ which has a smaller estimated distance to the target $11$. So that the path denoted by solid curved line can be found. Note that if decentralized search not only examines the edges that adjacent to current visited vertex, but also examines edges that are more than one hop away, shorter paths may be found. But due to the huge number of such edges, they have a relatively low possibility leading to a shorter path on average. Examining them may result in too much overhead. 

\subsection{Early termination}

An important question is that when performing decentralized search based on LCA distances, will the search eventually terminate? The answer is that as long as the source vertex $s$ and target vertex $t$ are reachable from each other, which in our case means that $s$ and $t$ have common landmarks, decentralized search will terminate in as much as $2 * max_{u,v}d_G(u,v)$ steps, where $max_{u,v}d_G(u,v)$ is the diameter of the network. Too see this, note that for the LCA distance of arbitrary source and target vertex $s$ and $t$, the following bound holds:

\begin{equation}
\label{equ:term}
\begin{split}
    d_{LCA}(s,t) \leq max_{l \in L}\{d_G(s,c_l(s,t)) + d_G(c_l(s,t),t)\} \\
		\leq max_{u,v}d_G(u,v) + max_{u,v}d_G(u,v)
\end{split}
\end{equation}

And at each step, as long as the current visited vertex is not the target vertex, there will always be a neighbor with a shorter LCA distance than current visited vertex to the target. So the estimated distance at each step will decrease at least by $1$. Therefore, the LCA distance of arbitrary pairs of vertices is bounded by $2 * max_{u,v}d_G(u,v)$ according to equation \ref{equ:term}. Decentralized search for arbitrary pairs of reachable vertices will terminate in as most $2 * max_{u,v}d_G(u,v)$ steps.

However, terminating only when the search reaches the target vertex is actually not an ideal stopping criterion, and will result in unnecessary overheads by examining redundant edges. Since the label of each vertex stores the shortest path of each vertex to each landmark, the shortest path follows the optimal substructure. That is, the path between any two vertices along the shortest path is also the shortest path of them. Considering that the label of each vertex follows the optimal substructure, the search can actually terminate once it reaches any vertex in the label of the target vertex. Because when the search reaching any vertex in the label of the target vertex, the label has already contained the shortest path from that vertex to the target vertex and decentralized search cannot find a path shorter than this one. Thus, the remaining path can be directly calculated from the label. Actually, decentralized search do not have to traverse vertices along the label of target vertices because these vertices have already been traversed by breadth first search for the same goal, which is reaching target vertex $u$, during preprocessing. With this stop criterion, the search space of decentralized search is significantly reduced and the search can terminate much earlier. The search will only visit at most the number of vertices from source vertex to the least common ancestor of source and target vertices. So the decentralized search can terminate in at most $max_{u,v}d_G(u,v)$ steps. We refer to this optimization as early termination.

\subsection{Bi-directional search}

\begin{figure}[t]
    \centering
    \includegraphics[width=\linewidth]{../figures/new_illustrate/bi_dec.pdf}
    \caption{Bidirectional decentralized search can explore different set of edges which may found path at different length. Searches start at vertex $7$ will lead to a path shorter than starting at $20$ by taking advantage of the edge $(5, 19)$.}
    \label{fig:bi_dec}
\end{figure}

As we have shown above, decentralized search is well suited for exploring the edges that connect currently visited vertex to a vertex that has shorter LCA distance to the target. Intuitively, edges that can lead to a shorter path from target vertex to the source vertex are equally important. By performing the decentralized search twice, one from source to target, and the other from target to source, we can achieve better accuracy by traversing more edges that have equal possibilities leading to shorter paths.

Unlike bidirectional BFS, which is aimed at finding shortest path when two search meet, decentralized search does not guarantee that two searches will meet each other. The only purpose for doing bidirectional decentralized search is to explore more edges that have a high possibility leading to a shorter path. When combining the results of bidirectional decentralized search, one can simply select the path with a shorter length. For example in Fig. \ref{fig:bi_dec}, the search starts from $20$ to $7$ can find a shorter path $p = (20, 17, 19, 5, 7)$ than the search starts at $7$. Due to that $0$ has a smaller LCA distance to $7$ than $19$, the edge $(19, 5)$ cannot be found by the search starts from $20$.

\subsection{Handle ties in decentralized search}

\begin{figure}[t]
    \centering
    \includegraphics[width=\linewidth]{../figures/new_illustrate/tie.pdf}
    \caption{Tie happens during decentralized search. Although LCA distances are the same, selecting different neighbor will search different part of the graph which may lead to paths at different length.}
    \label{fig:tie}
\end{figure}

Due to the small world property in complex networks, the shortest path tree of the landmark set usually has a limited depth. Furthermore, it is very common that a vertex connects to multiple vertices in a lower level of shortest path tree. So during decentralized search, there is a good chance that the search will encounter a tie at some point, that is, there are several neighbors of current visited vertex that have same shortest LCA distance to the target. For example in Fig. \ref{fig:tie}, to find path from $8$ to $6$, when traversing neighbors of vertex $8$, both vertex $11$ and $4$ have the same LCA distance to vertex $6$, but their actual distances to vertex $6$ are different due to edges currently invisible to the decentralized search. Labels of each neighbor, on the other hand, provide no clue which one can lead to a shorter path.

Since edges at two hops away are invisible to decentralized search, the search has no way to differentiate neighbors when a tie happens. Edges adjacent to these neighbor vertices have the same possibility leading to a shorter path based on information available to the decentralized search. To increase the possibility of finding shorter paths, multiple neighbors with same shortest LCA distance to the target can be selected as candidates for the next hop. In this way, the decentralized search will traverse multiple or even all neighbors with the same shortest LCA distance at the next step. Apparently, doing so will bring extra overheads due to the larger search space. Note that by selecting multiple neighbors, decentralized search can possibly find multiple paths with the same length. This increases the diversity of approximated path of decentralized search, which is required by some applications. 
\section{Greedy Index Construction}
\label{preprocessing}

This section describes the greedy index construction algorithm. For a vertex, all shortest paths from a landmark can be indexed as its label. We focus on the problem of deciding which shortest path to be indexed, so that better online query accuracy for average cases can be achieved. 

%\begin{figure*}[ht]
		%\vspace{-1cm}
    %\centering
    %\includegraphics[width=\linewidth]{./figures/new_illustrate/bfs_illustrate.pdf}
		%\vspace{-1cm}
    %\caption{Heuristic index construction pick shortest path with highest path degree during breadth first search}
    %\label{fig:bfs_illustrate}
		%\vspace{-5mm}
%\end{figure*}

As the core of decentralized search is to iteratively find neighbor vertices that have shortest LCA distances to the target. From the point of view of a vertex $u$, if the indexed shortest path intersects with many indexed shortest paths of other vertices, the possibility that other vertices have a small LCA distance to $u$ is going to be higher. With this intuition, we design our heuristic greedy index construction algorithm to store the shortest path with the highest ``centrality'', i.e., a path that intersects with most other shortest paths. To represent the ``centrality'' of a shortest path, we use the sum of vertex degrees along the path. %Although betweenness centrality fits our needs very well, its computation cost is too high~\cite{Riondato:2014:FAB:2556195.2556224}. Therefore, we use degrees as an alternative and refer to the sum of degrees of vertices along a path as path degree, denoted by $Pd$.

%Based on the path degree concept, our index construction procedure can be easily modified to index the shortest path with the highest path degree. 
As path degrees of shortest paths follow optimal substructures, 
%i.e., if a shortest path $(u, .., w, ..., v)$ has the highest path degree among all the shortest paths from $u$ to $v$, then the path degree of $(u, ..., w)$ is also the highest among all the shortest paths from $u$ to $w$.
to index the shortest path with the highest path degree, during BFS, suppose the search is visiting vertex $u$ and reach its neighbor $v$ with non empty $L(v)$, we perform a label update if $|L(v)| > |L(u)|$ and $Pd(u) + \rho(v) > Pd(v)$, where $\rho(v)$ denote the degree of vertex $v$. The detailed algorithm of greedy index construction is depicted in Algorithm~\ref{alg:index_construct}. 

Fig.~\ref{fig:bfs_illustrate} shows an example of how to greedily select the shortest path with the highest path degree during BFS. When traversing vertex $4$, even though vertex $8$ has already been indexed with a shortest path $(0, 1, 3, 8)$ into its label, due to that $(0, 2, 4, 8)$ has a higher path degree, the label of vertex $8$ is updated. The same happens to vertex $10$ while traversing vertex $6$. 

\begin{algorithm}[h]
    \caption{Greedy index construction on landmark $l$}
		\label{alg:index_construct}
    \begin{algorithmic}
				\Function{Index construction}{$l$}
						\State For each $v$ in $G$: $L(v) \gets \emptyset$
						\State For each $v$ in $G$: $Pd(v) \gets 0$
						\State $Q \gets \emptyset$
						\Comment {Queue for BFS}
						\State $L(l) = l$
						\State $Pd(l) = \rho(l)$ \
						\Comment {$\rho$ denotes degree}
						\State $Q.push(l)$
						\While{$Q \neq \emptyset$}
								\State $u = Q.pop()$
								\For{each $v_i$ adjecent to $u$}
										\If{$L(v_i) = \emptyset$}
												\State $L(v_i) = L(u)$
												\State append $v_i$ to $L(v_i)$
												\State $Pd(v_i) = Pd(u) + \rho(v_i)$
												\State $Q.push(v_i)$
										\ElsIf{$|L(v_i)| > |L(u)|$ {\bf and} $Pd(v_i) < Pd(u) + \rho(v_i)$}
												\State $L(v_i) = L(u)$
												\State append $v_i$ to $L(v_i)$
												\State $Pd(v_i) = Pd(u) + \rho(v_i)$
										\EndIf
								\EndFor
						\EndWhile
						\State \Return $L$
        \EndFunction
    \end{algorithmic}
\end{algorithm}

Note that if the landmark set is relatively large, then following the highest path degree heuristic may lead to redundant labels, i.e., similar indexed shortest path trees for multiple landmarks, which can compromise the accuracy of online searches. A simple way to solve this problem is to prioritize shortest paths which overlap less with shortest paths that have already been indexed.
\section{Distributed Implementations}
\label{implementation}

To handle extremely large graphs and large numbers of queries, we implement decentralized searches on a distributed general graph processing platform, Powergraph~\cite{180251}. As decentralized search has low online search space complexity and data dependencies upon each other, it is well suited to run multiple searches in a parallel way. 

\begin{figure*}[ht]
		\vspace{-0.7cm}
    \centering
    \includegraphics[width=\linewidth]{./figures/new_illustrate/system.pdf}
		\vspace{-1.2cm}
    \caption{A overview of distributed shortest path query processing system}
    \label{fig:system}
		\vspace{-2mm}
\end{figure*}

An overview of our shortest path query processing system is shown in Fig.~\ref{fig:system}. The system first use Powergraph to partition the graph onto multiple machines. Then several BFSs are performed to construct the index. After the index has been built, multiple shortest path queries can run in parallel. Large volumes of queries can submit repeatedly, for which responses will be generated at high throughput.

\subsection{Decentralized search vertex-program}

Decentralized search can be implemented as vertex-programs in Gather-Apply-Scatter model used by Powergraph. Indexes are stored in a distributed way as vertex data. Each query instance contains the approximated path and the label of target vertex, as it is not accessible on each machine locally. Each step of decentralized search is split into Gather, Apply and Scatter phases. In the Gather phase, the LCA distance to the target vertex $d_{LCA}$ is collected from each neighbor and accumulated with a $sum$ function by finding the neighbor with the smallest $d_{LCA}$ as a next step candidate. In the Apply phase, the candidate is appended to the approximated path $\tilde{p}$ and the termination condition is checked. If it is met, the result path will be recorded and the query will be terminated. Otherwise, the program will proceed to the Scatter phase to start a new vertex-program on the candidate vertex and pass on the query instance. Algorithm~\ref{alg:vc_dec} shows the detailed algorithm.

\begin{algorithm}
    \caption{Decentralized search vertex program on $u$}
		\label{alg:vc_dec}
    \begin{algorithmic}
        \Function{gather}{$L(v), L(t)$} 
						\Comment {on neighbor vertex $v$}
						\State \Return $d_{LCA}(v, t)$, $v$
				\EndFunction

        \Function{sum}{$d_{LCA}(v_1,t)$, $v_1$, $d_{LCA}(v_2,t)$, $v_2$}
						\If {$d_{LCA}(v_1,t) \le d_{LCA}(v_2,t)$}
								\State \Return $d_{LCA}(v_1,t), v_1$
						\Else
								\State \Return $d_{LCA}(v_2,t), v_2$
						\EndIf
				\EndFunction

        \Function{apply}{$L(t)$, $\tilde{p}(s,t)$, $d_{LCA}(v,t)$}
						\If {$v \in L(t)$}
								\State $p_{remain} \gets$ path from $v$ to $t$ in $L(t)$
								\State append $p_{remain}$ to $\tilde{p}(s,t)$
								\State store $\tilde{p}(s,t)$
								\State $termination = true$
						\Else
								\State append $v$ to $\tilde{p}(s,t)$
								\State $termination = false$
						\EndIf
				\EndFunction

        \Function{scatter}{$L(t)$, $\tilde{p}(s,t)$, $termination$}
						\If {$\neg termination$}
								\State Activate($v$, $L(t)$, $\tilde{p}(s,t)$)
						\EndIf
        \EndFunction
    \end{algorithmic}
\end{algorithm}
The communication for the decentralized search happens during the Gather and the Scatter phase. In the Gather phase, the label of target vertex need to be passed to multiple machines, and the size is $O(k{\sigma}_{max})$. Each Gather function returns a $d_{LCA}$ along side of its id. Therefore, only $O(k{\sigma}_{max})$ size of data is transferred in total. In the Scatter phase, communication happens when activating the next step candidate. The whole search instance, including approximated path and label of target vertex, needs to be transmitted. The total size is $O(k{\sigma}_{max})$. Since the search will take as much as $2{\sigma}_{max}$ steps. So the overall communication overhead for each query is $O(k{{\sigma}_{max}}^2)$. 

During decentralized search, only the approximated path $\tilde{p}$ is updated at each step. Therefore, there is only $RAR$ type of data dependency among multiple decentralized searches on the underlying graph. Depending on implementations, there may be output dependency, i.e. $WAW$, when output $\tilde{p}$.

%The low memory and communication cost, along side with $RAR$-only data dependency during the search, makes a large number of decentralized search very suitable to run in parallel. To modify the vertex program for parallel processing, each vertex program maintains a list of search instances. Each query can be processed independently for all three phase which can achieve very high level of parallelism.
%During the Gather phase, the label of target vertex for each query is transmitted to other machines, and $d_{LCA}$ is calculated for each query. Query are updated during the Apply phase. In the Scatter phase, each query is examined for whether activating a certain vertex or not.
%
\subsection{Distributed tie breaking strategy}
%According to tie breaking strategy, multiple neighbors may be returned as candidates at the Gather phase. In this case, the search instance clones itself into multiple search instances and appends each candidate on each of the search instance at the Apply phase, and multiple searches are activated at the Scatter phase. 
In our distributed implementation, the search instance clones itself into multiple search instances according to the tie breaking strategy. The problem is, as cloned search instances become independent in future steps, even one of them finds a shorter path than the others, it can hardly terminate other searches as such synchronizations are too costly in distributed environments. Therefore, a search may end up generating excessive number of child search instances. To overcome this problem, we pick one candidate at each step as a ``main'' candidate. For candidates that are not the ``main'' candidate, an extra termination condition is applied: In the following step, if the search cannot find a shorter path than expected, i.e. with a length shorter than $|\tilde{p}| + d_LCA$, the search will be discarded. 
%The search space can be effectively controlled without compromise much accuracy.

\subsection{Prune LCA computation}
A major part of the computation overhead of decentralized search is large number of LCA computations. It is possible to prune the number of LCA computation required at each step for decentralized search to reduce the overall computation overhead. Suppose the search is visiting vertex $u$, then $d_{LCA}(u, t)$ has already been calculated in previous steps. If a neighbor vertex $v$ is a child of $u$ on the indexed shortest path tree $SPT_l$, then the LCA computation for $v$ and $t$ on $SPT_l$ does not need to be carried on as $d_{LCA_l}(v, t) > d_{LCA_l}(u, t)$. In practice, this principle can prune almost half of the total number of LCA computations of a search on average.

\section{System Evaluation}
\label{evaluation}

\subsection{Evaluation Overview}
 
\subsection{Datasets}
 
\subsection{Methodology}
 
 
\subsection{Evaluation Results}
 
\section{Conclusion}
\label{conclusion}

In this paper, we describe a novel method to combine online and offline processing to allow approximate shortest path for extremely large graphs with high distance accuracy, path diversity and low overhead. 
%We demonstrate that different accuracy and overhead levels can be achieved by controlling the search space. 
We also develop an effective heuristic approach for constructing indexes that can improve the accuracy without increasing overhead. We implement our algorithm for cloud computing graph processing platforms, and demonstrate that our system can handle extremely large graphs and achieve high query processing throughput. The scalability of our system is good as both the number of machines and the number of parallel processed queries increases. 


% The following two commands are all you need in the
% initial runs of your .tex file to
% produce the bibliography for the citations in your paper.
\bibliographystyle{abbrv}
\bibliography{reference}  
% You must have a proper ".bib" file
%  and remember to run:
% latex bibtex latex latex
% to resolve all references
%
% ACM needs 'a single self-contained file'!
%
%APPENDICES are optional
%\balancecolumns

% That's all folks!
\end{document}
